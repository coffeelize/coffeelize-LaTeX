\documentclass{ctexart}
\usepackage{graphics}
\usepackage{listings}
\usepackage{coffeelize}
%\usepackage[sort&compress]{gbt7714}
%\bibliographystyle{gbt7714-numerical}

\begin{document}
%  ==== frontmatter ====
\title{Coffeelize}
\author{Coffeelize}
%%% 生成目录
%\tableofcontents
%%% 新开一页
%\clearpage

%  ====正文开始====

\section{写在开头}

\begin{lstlisting}
	//一行中可以同时声明多个变量
	int cnt = 0, res = 0;
	//if、else 中的语句若只有一条,可以不用写大括号
	if(nums[i]==1) ++cnt;
	else cnt = 0;
\end{lstlisting}

\begin{enumerate}
	\item 一行中可以同时声明多个变量
	\item 当if、else中的语句只有一条时,可以不用加大括号
	\item 对于for循环,其中的临时变量i也需要声明其类型,只不过一般都是int型
	\item 方法有返回值的话(非void),必然需要有return,否则编译报错
\end{enumerate}



\section{数组}

\subsection{27移除元素}

给你一个数组 nums 和一个值 val,你需要 原地 移除所有数值等于 val 的元素,并返回移除后数组的新长度。

\begin{lstlisting}
	class Solution {
		public int removeElement(int[] nums, int val) {
			int slow = 0;
			int fast = 0;
			for (fast = 0; fast < nums.length; fast++) {
				//快指针指向的不是目标元素
				if (nums[fast] != val) {
					// 1、将快指针指向的元素赋值给慢指针位置
					// 2、赋值完成后,slow++,慢指针向前移动一个位置
					nums[slow++] = nums[fast];
				}
			}
			return slow;
		}
	}
\end{lstlisting}

%\begin{note}
%	最大整数:Integer.MAX\_VALUE
%\end{note}



%  ====正文结束====
\end{document}